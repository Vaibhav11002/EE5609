\documentclass[journal,12pt,twocolumn]{IEEEtran}

\usepackage{setspace}
\usepackage{gensymb}

\singlespacing


\usepackage[cmex10]{amsmath}

\usepackage{amsthm}

\usepackage{mathrsfs}
\usepackage{txfonts}
\usepackage{stfloats}
\usepackage{bm}
\usepackage{cite}
\usepackage{cases}
\usepackage{subfig}

\usepackage{longtable}
\usepackage{multirow}

\usepackage{enumitem}
\usepackage{mathtools}
\usepackage{steinmetz}
\usepackage{tikz}
\usepackage{circuitikz}
\usepackage{verbatim}
\usepackage{tfrupee}
\usepackage[breaklinks=true]{hyperref}

\usepackage{tkz-euclide}

\usetikzlibrary{calc,math}
\usepackage{listings}
    \usepackage{color}                                            %%
    \usepackage{array}                                            %%
    \usepackage{longtable}                                        %%
    \usepackage{calc}                                             %%
    \usepackage{multirow}                                         %%
    \usepackage{hhline}                                           %%
    \usepackage{ifthen}                                           %%
    \usepackage{lscape}     
\usepackage{multicol}
\usepackage{chngcntr}

\DeclareMathOperator*{\Res}{Res}

\renewcommand\thesection{\arabic{section}}
\renewcommand\thesubsection{\thesection.\arabic{subsection}}
\renewcommand\thesubsubsection{\thesubsection.\arabic{subsubsection}}

\renewcommand\thesectiondis{\arabic{section}}
\renewcommand\thesubsectiondis{\thesectiondis.\arabic{subsection}}
\renewcommand\thesubsubsectiondis{\thesubsectiondis.\arabic{subsubsection}}


\hyphenation{op-tical net-works semi-conduc-tor}
\def\inputGnumericTable{}                                 %%

\lstset{
%language=C,
frame=single, 
breaklines=true,
columns=fullflexible
}
\begin{document}


\newtheorem{theorem}{Theorem}[section]
\newtheorem{problem}{Problem}
\newtheorem{proposition}{Proposition}[section]
\newtheorem{lemma}{Lemma}[section]
\newtheorem{corollary}[theorem]{Corollary}
\newtheorem{example}{Example}[section]
\newtheorem{definition}[problem]{Definition}

\newcommand{\BEQA}{\begin{eqnarray}}
\newcommand{\EEQA}{\end{eqnarray}}
\newcommand{\define}{\stackrel{\triangle}{=}}
\bibliographystyle{IEEEtran}
\providecommand{\mbf}{\mathbf}
\providecommand{\pr}[1]{\ensuremath{\Pr\left(#1\right)}}
\providecommand{\qfunc}[1]{\ensuremath{Q\left(#1\right)}}
\providecommand{\sbrak}[1]{\ensuremath{{}\left[#1\right]}}
\providecommand{\lsbrak}[1]{\ensuremath{{}\left[#1\right.}}
\providecommand{\rsbrak}[1]{\ensuremath{{}\left.#1\right]}}
\providecommand{\brak}[1]{\ensuremath{\left(#1\right)}}
\providecommand{\lbrak}[1]{\ensuremath{\left(#1\right.}}
\providecommand{\rbrak}[1]{\ensuremath{\left.#1\right)}}
\providecommand{\cbrak}[1]{\ensuremath{\left\{#1\right\}}}
\providecommand{\lcbrak}[1]{\ensuremath{\left\{#1\right.}}
\providecommand{\rcbrak}[1]{\ensuremath{\left.#1\right\}}}
\theoremstyle{remark}
\newtheorem{rem}{Remark}
\newcommand{\sgn}{\mathop{\mathrm{sgn}}}
\providecommand{\abs}[1]{\left\vert#1\right\vert}
\providecommand{\res}[1]{\Res\displaylimits_{#1}} 
\providecommand{\norm}[1]{\left\lVert#1\right\rVert}
%\providecommand{\norm}[1]{\lVert#1\rVert}
\providecommand{\mtx}[1]{\mathbf{#1}}
\providecommand{\mean}[1]{E\left[ #1 \right]}
\providecommand{\fourier}{\overset{\mathcal{F}}{ \rightleftharpoons}}
%\providecommand{\hilbert}{\overset{\mathcal{H}}{ \rightleftharpoons}}
\providecommand{\system}{\overset{\mathcal{H}}{ \longleftrightarrow}}
	%\newcommand{\solution}[2]{\textbf{Solution:}{#1}}
\newcommand{\solution}{\noindent \textbf{Solution: }}
\newcommand{\cosec}{\,\text{cosec}\,}
\providecommand{\dec}[2]{\ensuremath{\overset{#1}{\underset{#2}{\gtrless}}}}
\newcommand{\myvec}[1]{\ensuremath{\begin{pmatrix}#1\end{pmatrix}}}
\newcommand{\mydet}[1]{\ensuremath{\begin{vmatrix}#1\end{vmatrix}}}
\numberwithin{equation}{subsection}
\makeatletter
\@addtoreset{figure}{problem}
\makeatother
\let\StandardTheFigure\thefigure
\let\vec\mathbf
\renewcommand{\thefigure}{\theproblem}
\def\putbox#1#2#3{\makebox[0in][l]{\makebox[#1][l]{}\raisebox{\baselineskip}[0in][0in]{\raisebox{#2}[0in][0in]{#3}}}}
     \def\rightbox#1{\makebox[0in][r]{#1}}
     \def\centbox#1{\makebox[0in]{#1}}
     \def\topbox#1{\raisebox{-\baselineskip}[0in][0in]{#1}}
     \def\midbox#1{\raisebox{-0.5\baselineskip}[0in][0in]{#1}}
\vspace{3cm}
\title{Assignment 15}
\author{Gaydhane Vaibhav Digraj \\ Roll No. AI20MTECH11002}
\maketitle
\newpage
\bigskip
\renewcommand{\thefigure}{\theenumi}
\renewcommand{\thetable}{\theenumi}
\begin{abstract}
This document solves a problem involving linear transformations.
\end{abstract}
%
Download latex-tikz codes from 
%
\begin{lstlisting}
https://github.com/Vaibhav11002/EE5609/tree/master/Assignment_15
\end{lstlisting}
%
\section{Problem}
Let $\vec{V}$ be a vector space over the field $\vec{F}$ and $\vec{T}$ is a linear operator on $\vec{V}$. If $\vec{T}^2=0$, what can you say about the relation of the range of $\vec{T}$ to the null space of $\vec{T}$ ?
Give an example of linear operator $\vec{T}$ on $\vec{R}^2$ such that $\vec{T}^2=0$ but $\vec{T}\ne0$.

\section{Solution}
Given, 
\begin{align}
    \vec{T} : \vec{V} \xrightarrow{} \vec{V}
\end{align}
Now, $\vec{T}^2$ is also a linear operator on $\vec{R}^2$ as,
\begin{align}
    \vec{T}^2\brak{c\alpha} = \vec{T}\brak{\vec{T}\brak{c\alpha}} = \vec{T}\brak{c\vec{T}\brak{\alpha}} \\
    = c\vec{T}\brak{\vec{T}\brak{\alpha}} = c\vec{T}^2\brak{\alpha} 
\end{align}
Let some vector $\vec{y}\in $ Range$\brak{\vec{T}}$ then there exists $\vec{x} \in \vec{V}$ such that,
\begin{align}
    \vec{T}\brak{\vec{x}} = \vec{y}
\end{align}
Now given that,
\begin{align}
    \vec{T}^2(\vec{x}) = \vec{0} \\
    \implies \vec{T}(\vec{T}(\vec{x})) = \vec{0}\\
    \vec{T}(\vec{y}) = \vec{0}
\end{align}
$\therefore$ $\vec{y}$ lies in the Null space of $\vec{T}$. Hence $\vec{T}$ is singular. Thus, the range of $\vec{T}$ must be contained in Null space of $\vec{T}$ i.e.,
Range($\vec{T}$) $\subseteq$ NullSpace($\vec{T}$)

\subsection{Example}
\begin{align}
    \vec{T} : \vec{R}^2 \xrightarrow{} \vec{R}^2
\end{align}
Consider, 
\begin{align}
    \vec{T}\brak{\vec{x}} = \myvec{0&0\\1&0}\vec{x} \\
    \implies \vec{T} \ne 0 \label{t}
\end{align}
Now,
\begin{align}
    \vec{T}^2 : \vec{R}^2 \xrightarrow{} \vec{R}^2
\end{align}
\begin{align}
    &\vec{T}^2\brak{\vec{x}} = \vec{T}\brak{\vec{T}\brak{\vec{x}}} \\
    &= \myvec{0&0\\1&0}\myvec{0&0\\1&0}\vec{x} = \vec{0} \\
    &\implies \vec{T}^2\brak{\vec{x}} = \vec{0}
\end{align}
Thus $\vec{T}^2$ is a zero transformation,
\begin{align}
    \implies \vec{T}^2 = \vec{0} \label{t^2}
\end{align}
Thus from \eqref{t}, \eqref{t^2} it is clear that 
$\vec{T}^2 = 0$ but $\vec{T}\ne 0$.

\end{document}