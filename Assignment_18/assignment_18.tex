\documentclass[journal,12pt,twocolumn]{IEEEtran}

\usepackage{caption} 
\usepackage{setspace}
\usepackage{gensymb}

\singlespacing


\usepackage[cmex10]{amsmath}

\usepackage{amsthm}

\usepackage{mathrsfs}
\usepackage{txfonts}
\usepackage{stfloats}
\usepackage{bm}
\usepackage{cite}
\usepackage{cases}
\usepackage{subfig}

\usepackage{longtable}
\usepackage{multirow}

\usepackage{enumitem}
\usepackage{mathtools}
\usepackage{steinmetz}
\usepackage{tikz}
\usepackage{circuitikz}
\usepackage{verbatim}
\usepackage{tfrupee}
\usepackage[breaklinks=true]{hyperref}

\usepackage{tkz-euclide}

\usetikzlibrary{calc,math}
\usepackage{listings}
    \usepackage{color}                                            %%
    \usepackage{array}                                            %%
    \usepackage{longtable}                                        %%
    \usepackage{calc}                                             %%
    \usepackage{multirow}                                         %%
    \usepackage{hhline}                                           %%
    \usepackage{ifthen}                                           %%
    \usepackage{lscape}     
\usepackage{multicol}
\usepackage{chngcntr}

\DeclareMathOperator*{\Res}{Res}
\DeclareMathOperator{\range}{range}

\renewcommand\thesection{\arabic{section}}
\renewcommand\thesubsection{\thesection.\arabic{subsection}}
\renewcommand\thesubsubsection{\thesubsection.\arabic{subsubsection}}

\renewcommand\thesectiondis{\arabic{section}}
\renewcommand\thesubsectiondis{\thesectiondis.\arabic{subsection}}
\renewcommand\thesubsubsectiondis{\thesubsectiondis.\arabic{subsubsection}}


\hyphenation{op-tical net-works semi-conduc-tor}
\def\inputGnumericTable{}                                 %%

\lstset{
%language=C,
frame=single, 
breaklines=true,
columns=fullflexible
}
\begin{document}


\newtheorem{theorem}{Theorem}[section]
\newtheorem{problem}{Problem}
\newtheorem{proposition}{Proposition}[section]
\newtheorem{lemma}{Lemma}[section]
\newtheorem{corollary}[theorem]{Corollary}
\newtheorem{example}{Example}[section]
\newtheorem{definition}[problem]{Definition}

\newcommand{\BEQA}{\begin{eqnarray}}
\newcommand{\EEQA}{\end{eqnarray}}
\newcommand{\define}{\stackrel{\triangle}{=}}
\bibliographystyle{IEEEtran}
\providecommand{\mbf}{\mathbf}
\providecommand{\pr}[1]{\ensuremath{\Pr\left(#1\right)}}
\providecommand{\qfunc}[1]{\ensuremath{Q\left(#1\right)}}
\providecommand{\sbrak}[1]{\ensuremath{{}\left[#1\right]}}
\providecommand{\lsbrak}[1]{\ensuremath{{}\left[#1\right.}}
\providecommand{\rsbrak}[1]{\ensuremath{{}\left.#1\right]}}
\providecommand{\brak}[1]{\ensuremath{\left(#1\right)}}
\providecommand{\lbrak}[1]{\ensuremath{\left(#1\right.}}
\providecommand{\rbrak}[1]{\ensuremath{\left.#1\right)}}
\providecommand{\cbrak}[1]{\ensuremath{\left\{#1\right\}}}
\providecommand{\lcbrak}[1]{\ensuremath{\left\{#1\right.}}
\providecommand{\rcbrak}[1]{\ensuremath{\left.#1\right\}}}
\theoremstyle{remark}
\newtheorem{rem}{Remark}
\newcommand{\sgn}{\mathop{\mathrm{sgn}}}
\providecommand{\abs}[1]{\left\vert#1\right\vert}
\providecommand{\res}[1]{\Res\displaylimits_{#1}} 
\providecommand{\norm}[1]{\left\lVert#1\right\rVert}
%\providecommand{\norm}[1]{\lVert#1\rVert}
\providecommand{\mtx}[1]{\mathbf{#1}}
\providecommand{\mean}[1]{E\left[ #1 \right]}
\providecommand{\fourier}{\overset{\mathcal{F}}{ \rightleftharpoons}}
%\providecommand{\hilbert}{\overset{\mathcal{H}}{ \rightleftharpoons}}
\providecommand{\system}{\overset{\mathcal{H}}{ \longleftrightarrow}}
	%\newcommand{\solution}[2]{\textbf{Solution:}{#1}}
\newcommand{\solution}{\noindent \textbf{Solution: }}
\newcommand{\cosec}{\,\text{cosec}\,}
\providecommand{\dec}[2]{\ensuremath{\overset{#1}{\underset{#2}{\gtrless}}}}
\newcommand{\myvec}[1]{\ensuremath{\begin{pmatrix}#1\end{pmatrix}}}
\newcommand{\mydet}[1]{\ensuremath{\begin{vmatrix}#1\end{vmatrix}}}
\numberwithin{equation}{subsection}
\makeatletter
\@addtoreset{figure}{problem}
\makeatother
\let\StandardTheFigure\thefigure
\let\vec\mathbf
\renewcommand{\thefigure}{\theproblem}
\def\putbox#1#2#3{\makebox[0in][l]{\makebox[#1][l]{}\raisebox{\baselineskip}[0in][0in]{\raisebox{#2}[0in][0in]{#3}}}}
     \def\rightbox#1{\makebox[0in][r]{#1}}
     \def\centbox#1{\makebox[0in]{#1}}
     \def\topbox#1{\raisebox{-\baselineskip}[0in][0in]{#1}}
     \def\midbox#1{\raisebox{-0.5\baselineskip}[0in][0in]{#1}}
\vspace{3cm}
\title{Assignment 18}
\author{Gaydhane Vaibhav Digraj \\ Roll No. AI20MTECH11002}
\maketitle
\newpage
\bigskip
\renewcommand{\thefigure}{\theenumi}
\renewcommand{\thetable}{\theenumi}
\begin{abstract}
This document solves a problem based on polynomial vector spaces.
\end{abstract}
%
Download latex-tikz codes from 
%
\begin{lstlisting}
https://github.com/Vaibhav11002/EE5609/tree/master/Assignment_18
\end{lstlisting}
%
\section{Problem}
Consider the vector space V of real polynomials of degree less than or equal to n. Fix distinct real numbers $a_0, a_1, \cdots, a_k$. For $p \in V$
\begin{align*}
    max\cbrak{\abs{p(a_j)}: 0\leq j \leq k}
\end{align*}
defines a norm on V
\begin{enumerate}
    \item only if $k<n$
    \item only if $k\ge n$
    \item if $ k+1\leq n$ 
    \item if $k \ge n+1$
\end{enumerate}

\section{Solution}
Options 2 and 4 are correct as verified in the table \ref{table2}
\renewcommand{\thetable}{1}
\begin{table*}[ht!]
\begin{center}
\begin{tabular}{|c|c|}
\hline
\textbf{Properties}&\textbf{Norm $\forall x \in V$}\\
\hline
Positivity & $\norm{x}\ge 0, \norm{x} = 0 \iff x=0 $ \\
\hline
Scalar Multiplication & $\norm{\alpha x} = \abs{\alpha}\norm{x}, \alpha \in F $\\
\hline
Triangle Inequality & $\norm{x+y} \le \norm{x} + \norm{y} $\\
\hline
\end{tabular}
\caption{Properties of Norm}
\label{table1}
\end{center}
\end{table*}

\renewcommand{\thetable}{2}
\begin{table*}[ht!]
\begin{center}
\begin{tabular}{|c|c|}
\hline
\multicolumn{2}{|c|}{
For $p \in V$ then the norm, 
$max\cbrak{\abs{p(a_j)}: 0 \leq j \leq k}=0 \iff \abs{p(a_j)}_{0 \leq j \leq k}=0$
} \\[3ex]
\hline
\textbf{Conditions} & \textbf{Explanation} \\[0.5ex]
\hline
\text{only if $k < n$} & 
A polynomial doesn't necessarily have $k$ distinct real roots,\\
&i.e., it may have repeated, complex roots. \\
Example:& let $p$ be polynomial of degree $n=2$ and $k=1$ given by:-\\
&  \parbox{12cm}{\begin{align}
    p(x) &= x^2 + 4x + 4 \\
    %p(x) &=0 \implies x=-2, -2 \\
    \abs{p(a_j)}_{0\le j \le 1} &= 0 \implies a_0 = -2, a_1 = -2
\end{align}}\\ 
& but $a_0, a_1, \cdots, a_k$ should be distinct real numbers.\\
& This contradicts the property of Norm. Thus condition fails.
\\ [0.5ex]
\hline
\text{only if $k\ge n$} & 
p is a polynomial of degree $\le$n,\\
& it can't have more than $n$ roots and is only possible when,\\
&$p(x)=0 \implies \abs{p(a_j)}_{0 \leq j \leq k}=0$\\
& hence $p$ is identically zero. Thus condition satisfies.
\\ [0.5ex]
\hline
\text{if $k+1 \leq n$} & 
Not a norm for $k<n$. Hence incorrect. 
\\ [0.5ex]
\hline
\text{if $k \ge n+1$} &
Norm for $k \ge n$. Hence correct.
\\[0.5ex]
\hline
\end{tabular}
\caption{Verifying Positivity Property of Norm}
\label{table2}
\end{center}
\end{table*}

\section{Example}
The scalar multiplication and triangle inequality properties holds true for all $k$.
\begin{align}
    &max\cbrak{\abs{\alpha p(a_j)}} = \abs{\alpha}max\cbrak{\abs{p(a_j)}}\\
    &max\cbrak{\abs{p(a_i)+p(a_j)}} \le max\cbrak{\abs{p(a_i)}} + max\cbrak{\abs{p(a_j)}}
\end{align}
The positivity property holds true only if $k \ge n$ as more than $n$ roots are possible when, 
\begin{align}
    p(x) &= 0 \implies \abs{p(a_j)}_{0 \leq j \leq k}=0 
\end{align}
\begin{align}
    \implies max\cbrak{\abs{p(a_j)}: 0 \leq j \leq k}=0
\end{align}



\end{document}