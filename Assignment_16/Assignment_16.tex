\documentclass[journal,12pt,twocolumn]{IEEEtran}

\usepackage{setspace}
\usepackage{gensymb}

\singlespacing


\usepackage[cmex10]{amsmath}

\usepackage{amsthm}

\usepackage{mathrsfs}
\usepackage{txfonts}
\usepackage{stfloats}
\usepackage{bm}
\usepackage{cite}
\usepackage{cases}
\usepackage{subfig}

\usepackage{longtable}
\usepackage{multirow}

\usepackage{enumitem}
\usepackage{mathtools}
\usepackage{steinmetz}
\usepackage{tikz}
\usepackage{circuitikz}
\usepackage{verbatim}
\usepackage{tfrupee}
\usepackage[breaklinks=true]{hyperref}

\usepackage{tkz-euclide}

\usetikzlibrary{calc,math}
\usepackage{listings}
    \usepackage{color}                                            %%
    \usepackage{array}                                            %%
    \usepackage{longtable}                                        %%
    \usepackage{calc}                                             %%
    \usepackage{multirow}                                         %%
    \usepackage{hhline}                                           %%
    \usepackage{ifthen}                                           %%
    \usepackage{lscape}     
\usepackage{multicol}
\usepackage{chngcntr}

\DeclareMathOperator*{\Res}{Res}

\renewcommand\thesection{\arabic{section}}
\renewcommand\thesubsection{\thesection.\arabic{subsection}}
\renewcommand\thesubsubsection{\thesubsection.\arabic{subsubsection}}

\renewcommand\thesectiondis{\arabic{section}}
\renewcommand\thesubsectiondis{\thesectiondis.\arabic{subsection}}
\renewcommand\thesubsubsectiondis{\thesubsectiondis.\arabic{subsubsection}}


\hyphenation{op-tical net-works semi-conduc-tor}
\def\inputGnumericTable{}                                 %%

\lstset{
%language=C,
frame=single, 
breaklines=true,
columns=fullflexible
}
\begin{document}


\newtheorem{theorem}{Theorem}[section]
\newtheorem{problem}{Problem}
\newtheorem{proposition}{Proposition}[section]
\newtheorem{lemma}{Lemma}[section]
\newtheorem{corollary}[theorem]{Corollary}
\newtheorem{example}{Example}[section]
\newtheorem{definition}[problem]{Definition}

\newcommand{\BEQA}{\begin{eqnarray}}
\newcommand{\EEQA}{\end{eqnarray}}
\newcommand{\define}{\stackrel{\triangle}{=}}
\bibliographystyle{IEEEtran}
\providecommand{\mbf}{\mathbf}
\providecommand{\pr}[1]{\ensuremath{\Pr\left(#1\right)}}
\providecommand{\qfunc}[1]{\ensuremath{Q\left(#1\right)}}
\providecommand{\sbrak}[1]{\ensuremath{{}\left[#1\right]}}
\providecommand{\lsbrak}[1]{\ensuremath{{}\left[#1\right.}}
\providecommand{\rsbrak}[1]{\ensuremath{{}\left.#1\right]}}
\providecommand{\brak}[1]{\ensuremath{\left(#1\right)}}
\providecommand{\lbrak}[1]{\ensuremath{\left(#1\right.}}
\providecommand{\rbrak}[1]{\ensuremath{\left.#1\right)}}
\providecommand{\cbrak}[1]{\ensuremath{\left\{#1\right\}}}
\providecommand{\lcbrak}[1]{\ensuremath{\left\{#1\right.}}
\providecommand{\rcbrak}[1]{\ensuremath{\left.#1\right\}}}
\theoremstyle{remark}
\newtheorem{rem}{Remark}
\newcommand{\sgn}{\mathop{\mathrm{sgn}}}
\providecommand{\abs}[1]{\left\vert#1\right\vert}
\providecommand{\res}[1]{\Res\displaylimits_{#1}} 
\providecommand{\norm}[1]{\left\lVert#1\right\rVert}
%\providecommand{\norm}[1]{\lVert#1\rVert}
\providecommand{\mtx}[1]{\mathbf{#1}}
\providecommand{\mean}[1]{E\left[ #1 \right]}
\providecommand{\fourier}{\overset{\mathcal{F}}{ \rightleftharpoons}}
%\providecommand{\hilbert}{\overset{\mathcal{H}}{ \rightleftharpoons}}
\providecommand{\system}{\overset{\mathcal{H}}{ \longleftrightarrow}}
	%\newcommand{\solution}[2]{\textbf{Solution:}{#1}}
\newcommand{\solution}{\noindent \textbf{Solution: }}
\newcommand{\cosec}{\,\text{cosec}\,}
\providecommand{\dec}[2]{\ensuremath{\overset{#1}{\underset{#2}{\gtrless}}}}
\newcommand{\myvec}[1]{\ensuremath{\begin{pmatrix}#1\end{pmatrix}}}
\newcommand{\mydet}[1]{\ensuremath{\begin{vmatrix}#1\end{vmatrix}}}
\numberwithin{equation}{subsection}
\makeatletter
\@addtoreset{figure}{problem}
\makeatother
\let\StandardTheFigure\thefigure
\let\vec\mathbf
\renewcommand{\thefigure}{\theproblem}
\def\putbox#1#2#3{\makebox[0in][l]{\makebox[#1][l]{}\raisebox{\baselineskip}[0in][0in]{\raisebox{#2}[0in][0in]{#3}}}}
     \def\rightbox#1{\makebox[0in][r]{#1}}
     \def\centbox#1{\makebox[0in]{#1}}
     \def\topbox#1{\raisebox{-\baselineskip}[0in][0in]{#1}}
     \def\midbox#1{\raisebox{-0.5\baselineskip}[0in][0in]{#1}}
\vspace{3cm}
\title{Assignment 15}
\author{Gaydhane Vaibhav Digraj \\ Roll No. AI20MTECH11002}
\maketitle
\newpage
\bigskip
\renewcommand{\thefigure}{\theenumi}
\renewcommand{\thetable}{\theenumi}
\begin{abstract}
This document solves a problem involving ordered basis of linear transformation.
\end{abstract}
%
Download latex-tikz codes from 
%
\begin{lstlisting}
https://github.com/Vaibhav11002/EE5609/tree/master/Assignment_16
\end{lstlisting}
%
\section{Problem}
Let V be an n-dimensional vector space over the field F, and let $\mathcal{B} = \cbrak{\alpha_1,\ldots, \alpha_n}$ be an ordered basis for V then there is a unique linear operator T on V such that 
\begin{align*}
   &T\alpha_{j} = \alpha_{j+1},  j=1,\ldots,n-1 \\
   &T\alpha_n = 0. 
\end{align*}
What is the matrix A of T in the ordered basis $\mathcal{B}$?

\section{Coordinates of a Vector}
Let $\mathcal{B} = \cbrak{\vec{v}_1, \vec{v}_2,\ldots,\vec{v}_n}$ be the ordered basis of an n-dimensional vector space V over field F and let $\vec{v}\in V$. If
\begin{align}
    \vec{v} = \beta_{1}\vec{v}_1 + \beta_{2}\vec{v}_2 + \ldots + \beta_{n}\vec{v}_n \label{v}
\end{align}
then the tuple $(\beta_{1},\beta_{2},\ldots,\beta_{n})$ is called the coordinate of the vector $\vec{v}$ with respect to the ordered basis $\mathcal{B}$. 
It is denoted by the column vector,
\begin{align}
    [\vec{v}]_\mathcal{B} = (\beta_{1},\beta_{2},\ldots,\beta_{n})^{T} \label{v_col}
\end{align}

\section{Solution}
Given that, 
\begin{align}
    T: V \xrightarrow{} V \\
    [T(\alpha)]_\mathcal{B} = A[\alpha]_{\mathcal{B}} \\
    T\alpha_{j} = \alpha_{j+1} \\
    T\alpha_{n} = 0
\end{align}
where $j = 1,\ldots, n-1$. 
The matrix A of T in the ordered basis $\mathcal{B}$ is given by, 
\begin{align}
    \implies A &= \myvec{[T\alpha_1]_{\mathcal{B}} & \cdots & [T\alpha_n]_{\mathcal{B}}} \label{A_basis}
\end{align}
For $j = 1,\ldots, n-1$ we have,
\begin{align}
    T\alpha_{j} = \alpha_{j+1}
\end{align}
From \eqref{v}, \eqref{v_col} we can write, 
\begin{align}
    &T\alpha_{j} = 0\alpha_{1} + \ldots + 0\alpha_{j} + 1\alpha_{j+1} + \ldots + 0\alpha_{n} \\
    &\implies [T\alpha_{j}]_{\mathcal{B}} = (0,\ldots,0, 1,0,\ldots, 0)^{T} \label{talpha_j}
\end{align}
where 1 is in ($j+1$)th position. Now,
\begin{align}
    T\alpha_{n} &= 0 \\
    \implies [T\alpha_{n}]_{\mathcal{B}} &= 0 \label{talpha_n}
\end{align}
Thus from \eqref{A_basis}, \eqref{talpha_j} and \eqref{talpha_n} we get matrix A in the ordered basis $\mathcal{B}$ as,
\begin{align}
    A = \myvec{0&0&0&0&\cdots&0&0& \\ 1&0&0&0&\cdots&0&0 \\ 0&1&0&0&\cdots&0&0 \\ 0&0&1&0&\cdots&0&0 \\ \vdots&\vdots&\vdots&\vdots&\ddots&\vdots&\vdots \\ 0&0&0&0&\cdots&1&0}
\end{align}

\end{document}