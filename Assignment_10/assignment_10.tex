\documentclass[journal,12pt,twocolumn]{IEEEtran}

\usepackage{setspace}
\usepackage{gensymb}

\singlespacing


\usepackage[cmex10]{amsmath}

\usepackage{amsthm}

\usepackage{mathrsfs}
\usepackage{txfonts}
\usepackage{stfloats}
\usepackage{bm}
\usepackage{cite}
\usepackage{cases}
\usepackage{subfig}

\usepackage{longtable}
\usepackage{multirow}

\usepackage{enumitem}
\usepackage{mathtools}
\usepackage{steinmetz}
\usepackage{tikz}
\usepackage{circuitikz}
\usepackage{verbatim}
\usepackage{tfrupee}
\usepackage[breaklinks=true]{hyperref}

\usepackage{tkz-euclide}

\usetikzlibrary{calc,math}
\usepackage{listings}
    \usepackage{color}                                            %%
    \usepackage{array}                                            %%
    \usepackage{longtable}                                        %%
    \usepackage{calc}                                             %%
    \usepackage{multirow}                                         %%
    \usepackage{hhline}                                           %%
    \usepackage{ifthen}                                           %%
    \usepackage{lscape}     
\usepackage{multicol}
\usepackage{chngcntr}

\DeclareMathOperator*{\Res}{Res}

\renewcommand\thesection{\arabic{section}}
\renewcommand\thesubsection{\thesection.\arabic{subsection}}
\renewcommand\thesubsubsection{\thesubsection.\arabic{subsubsection}}

\renewcommand\thesectiondis{\arabic{section}}
\renewcommand\thesubsectiondis{\thesectiondis.\arabic{subsection}}
\renewcommand\thesubsubsectiondis{\thesubsectiondis.\arabic{subsubsection}}


\hyphenation{op-tical net-works semi-conduc-tor}
\def\inputGnumericTable{}                                 %%

\lstset{
%language=C,
frame=single, 
breaklines=true,
columns=fullflexible
}
\begin{document}


\newtheorem{theorem}{Theorem}[section]
\newtheorem{problem}{Problem}
\newtheorem{proposition}{Proposition}[section]
\newtheorem{lemma}{Lemma}[section]
\newtheorem{corollary}[theorem]{Corollary}
\newtheorem{example}{Example}[section]
\newtheorem{definition}[problem]{Definition}

\newcommand{\BEQA}{\begin{eqnarray}}
\newcommand{\EEQA}{\end{eqnarray}}
\newcommand{\define}{\stackrel{\triangle}{=}}
\bibliographystyle{IEEEtran}
\providecommand{\mbf}{\mathbf}
\providecommand{\pr}[1]{\ensuremath{\Pr\left(#1\right)}}
\providecommand{\qfunc}[1]{\ensuremath{Q\left(#1\right)}}
\providecommand{\sbrak}[1]{\ensuremath{{}\left[#1\right]}}
\providecommand{\lsbrak}[1]{\ensuremath{{}\left[#1\right.}}
\providecommand{\rsbrak}[1]{\ensuremath{{}\left.#1\right]}}
\providecommand{\brak}[1]{\ensuremath{\left(#1\right)}}
\providecommand{\lbrak}[1]{\ensuremath{\left(#1\right.}}
\providecommand{\rbrak}[1]{\ensuremath{\left.#1\right)}}
\providecommand{\cbrak}[1]{\ensuremath{\left\{#1\right\}}}
\providecommand{\lcbrak}[1]{\ensuremath{\left\{#1\right.}}
\providecommand{\rcbrak}[1]{\ensuremath{\left.#1\right\}}}
\theoremstyle{remark}
\newtheorem{rem}{Remark}
\newcommand{\sgn}{\mathop{\mathrm{sgn}}}
\providecommand{\abs}[1]{\left\vert#1\right\vert}
\providecommand{\res}[1]{\Res\displaylimits_{#1}} 
\providecommand{\norm}[1]{\left\lVert#1\right\rVert}
%\providecommand{\norm}[1]{\lVert#1\rVert}
\providecommand{\mtx}[1]{\mathbf{#1}}
\providecommand{\mean}[1]{E\left[ #1 \right]}
\providecommand{\fourier}{\overset{\mathcal{F}}{ \rightleftharpoons}}
%\providecommand{\hilbert}{\overset{\mathcal{H}}{ \rightleftharpoons}}
\providecommand{\system}{\overset{\mathcal{H}}{ \longleftrightarrow}}
	%\newcommand{\solution}[2]{\textbf{Solution:}{#1}}
\newcommand{\solution}{\noindent \textbf{Solution: }}
\newcommand{\cosec}{\,\text{cosec}\,}
\providecommand{\dec}[2]{\ensuremath{\overset{#1}{\underset{#2}{\gtrless}}}}
\newcommand{\myvec}[1]{\ensuremath{\begin{pmatrix}#1\end{pmatrix}}}
\newcommand{\mydet}[1]{\ensuremath{\begin{vmatrix}#1\end{vmatrix}}}
\numberwithin{equation}{subsection}
\makeatletter
\@addtoreset{figure}{problem}
\makeatother
\let\StandardTheFigure\thefigure
\let\vec\mathbf
\renewcommand{\thefigure}{\theproblem}
\def\putbox#1#2#3{\makebox[0in][l]{\makebox[#1][l]{}\raisebox{\baselineskip}[0in][0in]{\raisebox{#2}[0in][0in]{#3}}}}
     \def\rightbox#1{\makebox[0in][r]{#1}}
     \def\centbox#1{\makebox[0in]{#1}}
     \def\topbox#1{\raisebox{-\baselineskip}[0in][0in]{#1}}
     \def\midbox#1{\raisebox{-0.5\baselineskip}[0in][0in]{#1}}
\vspace{3cm}
\title{Assignment 10}
\author{Gaydhane Vaibhav Digraj \\ Roll No. AI20MTECH11002}
\maketitle
\newpage
\bigskip
\renewcommand{\thefigure}{\theenumi}
\renewcommand{\thetable}{\theenumi}
\begin{abstract}
This document solves a problem involving matrix multiplication.
\end{abstract}
%
Download latex-tikz codes from 
%
\begin{lstlisting}
https://github.com/Vaibhav11002/EE5609/tree/master/Assignment_10
\end{lstlisting}
%
\section{Problem}
Find two different 2$\times$2 matrices $\vec{A}$ such that 

$\vec{A}^{2}=0$ but $\vec{A}\ne0$

\section{Solution}
The matrix $\vec{A}$ can be given by, 
\begin{align}
    \vec{A} = \myvec{\vec{m}&\vec{n}} \\
    \vec{m} = \myvec{m_1\\m_2}, \vec{n} = \myvec{n_1\\n_2}
\end{align}
Now, 
\begin{align}
    \vec{A}^{2} &= \vec{A}\vec{A} = 0 \\
    \implies\vec{A}^{2} &= \myvec{m_1&n_1\\m_2&n_2}\myvec{\vec{m}&\vec{n}} = \myvec{\vec{c_1}&\vec{c_2}} = 0 \label{A2}
\end{align}
From \eqref{A2}, we say that the columns of matrix $\vec{A}$ must lie in the null space of $\vec{A}$. Also atleast one of the columns must be non-zero since given $\vec{A}\ne0$. 
Thus, the null space of $\vec{A}$ contains non zero vectors, hence $rank(\vec{A})<2$. 
This implies that the columns of $\vec{A}$ are linearly dependent.
Hence, $\vec{A}$ is a singular matrix. 
\begin{align}
    \vec{c_1} &= m_1\myvec{m_1\\m_2} + m_2\myvec{n_1\\n_2} = 0 \\
    \vec{c_2} &= n_1\myvec{m_1\\m_2} + n_2\myvec{n_1\\n_2} = 0
\end{align}
%Since $\vec{A}\ne0$, we say $\vec{m}=\vec{n}\ne0$
Let's assume, 
\begin{align}
    \vec{m} &= \myvec{a\\a}
\end{align}
then, 
\begin{align}
    &\vec{c_1} = \myvec{a^2+an_1\\a^2+an_2}=0
\end{align}
thus $a=0$ or $n_1=n_2=-a$. When $a=0$, 
\begin{align}
    &\implies\vec{c_2} = \myvec{n_1n_2\\n_2^2}=0\\
    &\implies n_2 = 0
\end{align}
Thus we get, 
\begin{align}
    \vec{A} &= \myvec{a&-a\\a&-a} ; a\ne0
\end{align}
and
\begin{align}
    \vec{A} &= \myvec{0&n_1\\0&0} ; n_1\ne0
\end{align}
Now, let's assume, 
\begin{align}
    \vec{n} &= \myvec{a\\a}
\end{align}
then, 
\begin{align}
    \vec{c_2} &= \myvec{am_1+a^2\\am_2+a^2} = 0
\end{align}
thus $a=0$ or $m_1=m_2=-a$. When $a=0$, 
\begin{align}
    &\implies\vec{c_1} = \myvec{m_1^2\\m_1m_2}=0\\
    &\implies m_1=0
\end{align}
Thus we get, 
\begin{align}
    \vec{A} &= \myvec{-a&a\\-a&a} ; a\ne0
\end{align}
and 
\begin{align}
    \vec{A} &= \myvec{0&0\\m_2&0} ; m_2\ne0
\end{align}
One such matrix also is, 
\begin{align}
    \vec{A} &= \myvec{a&a\\-a&-a} ; a\ne0
\end{align}
Thus two different 2$\times2$ matrices are, 
\begin{align}
    \vec{A}&=\myvec{0&1\\0&0}\\
    \vec{A}&=\myvec{2&-2\\2&-2}
\end{align}

\end{document}