\documentclass[journal,12pt,twocolumn]{IEEEtran}

\usepackage{setspace}
\usepackage{gensymb}

\singlespacing


\usepackage[cmex10]{amsmath}

\usepackage{amsthm}

\usepackage{mathrsfs}
\usepackage{txfonts}
\usepackage{stfloats}
\usepackage{bm}
\usepackage{cite}
\usepackage{cases}
\usepackage{subfig}

\usepackage{longtable}
\usepackage{multirow}

\usepackage{enumitem}
\usepackage{mathtools}
\usepackage{steinmetz}
\usepackage{tikz}
\usepackage{circuitikz}
\usepackage{verbatim}
\usepackage{tfrupee}
\usepackage[breaklinks=true]{hyperref}

\usepackage{tkz-euclide}

\usetikzlibrary{calc,math}
\usepackage{listings}
    \usepackage{color}                                            %%
    \usepackage{array}                                            %%
    \usepackage{longtable}                                        %%
    \usepackage{calc}                                             %%
    \usepackage{multirow}                                         %%
    \usepackage{hhline}                                           %%
    \usepackage{ifthen}                                           %%
    \usepackage{lscape}     
\usepackage{multicol}
\usepackage{chngcntr}

\DeclareMathOperator*{\Res}{Res}

\renewcommand\thesection{\arabic{section}}
\renewcommand\thesubsection{\thesection.\arabic{subsection}}
\renewcommand\thesubsubsection{\thesubsection.\arabic{subsubsection}}

\renewcommand\thesectiondis{\arabic{section}}
\renewcommand\thesubsectiondis{\thesectiondis.\arabic{subsection}}
\renewcommand\thesubsubsectiondis{\thesubsectiondis.\arabic{subsubsection}}


\hyphenation{op-tical net-works semi-conduc-tor}
\def\inputGnumericTable{}                                 %%

\lstset{
%language=C,
frame=single, 
breaklines=true,
columns=fullflexible
}
\begin{document}


\newtheorem{theorem}{Theorem}[section]
\newtheorem{problem}{Problem}
\newtheorem{proposition}{Proposition}[section]
\newtheorem{lemma}{Lemma}[section]
\newtheorem{corollary}[theorem]{Corollary}
\newtheorem{example}{Example}[section]
\newtheorem{definition}[problem]{Definition}

\newcommand{\BEQA}{\begin{eqnarray}}
\newcommand{\EEQA}{\end{eqnarray}}
\newcommand{\define}{\stackrel{\triangle}{=}}
\bibliographystyle{IEEEtran}
\providecommand{\mbf}{\mathbf}
\providecommand{\pr}[1]{\ensuremath{\Pr\left(#1\right)}}
\providecommand{\qfunc}[1]{\ensuremath{Q\left(#1\right)}}
\providecommand{\sbrak}[1]{\ensuremath{{}\left[#1\right]}}
\providecommand{\lsbrak}[1]{\ensuremath{{}\left[#1\right.}}
\providecommand{\rsbrak}[1]{\ensuremath{{}\left.#1\right]}}
\providecommand{\brak}[1]{\ensuremath{\left(#1\right)}}
\providecommand{\lbrak}[1]{\ensuremath{\left(#1\right.}}
\providecommand{\rbrak}[1]{\ensuremath{\left.#1\right)}}
\providecommand{\cbrak}[1]{\ensuremath{\left\{#1\right\}}}
\providecommand{\lcbrak}[1]{\ensuremath{\left\{#1\right.}}
\providecommand{\rcbrak}[1]{\ensuremath{\left.#1\right\}}}
\theoremstyle{remark}
\newtheorem{rem}{Remark}
\newcommand{\sgn}{\mathop{\mathrm{sgn}}}
\providecommand{\abs}[1]{\left\vert#1\right\vert}
\providecommand{\res}[1]{\Res\displaylimits_{#1}} 
\providecommand{\norm}[1]{\left\lVert#1\right\rVert}
%\providecommand{\norm}[1]{\lVert#1\rVert}
\providecommand{\mtx}[1]{\mathbf{#1}}
\providecommand{\mean}[1]{E\left[ #1 \right]}
\providecommand{\fourier}{\overset{\mathcal{F}}{ \rightleftharpoons}}
%\providecommand{\hilbert}{\overset{\mathcal{H}}{ \rightleftharpoons}}
\providecommand{\system}{\overset{\mathcal{H}}{ \longleftrightarrow}}
	%\newcommand{\solution}[2]{\textbf{Solution:}{#1}}
\newcommand{\solution}{\noindent \textbf{Solution: }}
\newcommand{\cosec}{\,\text{cosec}\,}
\providecommand{\dec}[2]{\ensuremath{\overset{#1}{\underset{#2}{\gtrless}}}}
\newcommand{\myvec}[1]{\ensuremath{\begin{pmatrix}#1\end{pmatrix}}}
\newcommand{\mydet}[1]{\ensuremath{\begin{vmatrix}#1\end{vmatrix}}}
\numberwithin{equation}{subsection}
\makeatletter
\@addtoreset{figure}{problem}
\makeatother
\let\StandardTheFigure\thefigure
\let\vec\mathbf
\renewcommand{\thefigure}{\theproblem}
\def\putbox#1#2#3{\makebox[0in][l]{\makebox[#1][l]{}\raisebox{\baselineskip}[0in][0in]{\raisebox{#2}[0in][0in]{#3}}}}
     \def\rightbox#1{\makebox[0in][r]{#1}}
     \def\centbox#1{\makebox[0in]{#1}}
     \def\topbox#1{\raisebox{-\baselineskip}[0in][0in]{#1}}
     \def\midbox#1{\raisebox{-0.5\baselineskip}[0in][0in]{#1}}
\vspace{3cm}
\title{Assignment 17}
\author{Gaydhane Vaibhav Digraj \\ Roll No. AI20MTECH11002}
\maketitle
\newpage
\bigskip
\renewcommand{\thefigure}{\theenumi}
\renewcommand{\thetable}{\theenumi}
\begin{abstract}
This document solves a problem involving linear functional.
\end{abstract}
%
Download latex-tikz codes from 
%
\begin{lstlisting}
https://github.com/Vaibhav11002/EE5609/tree/master/Assignment_17
\end{lstlisting}
%
\section{Problem}
Similar matrices have the same trace. Thus we can define the trace of a linear operator on a finite-dimensional space to the trace of any matrix which represents the operator in a ordered basis. This is well-defined since all such representing matrices for one operator are similar. 

Now let V be the space of all 2$\times$2 matrices over the field F and let P be a fixed 2$\times$2 matrix. Let T be the linear operator on V defined by $T(A)=PA$. Prove that $trace(T)=2trace(P)$. 

\section{Solution}
Given V is the space of all 2$\times$2 matrices over field F. P is a 2$\times$2 matrix, 
\begin{align}
    P = \myvec{p_{11}&p_{12}\\ p_{21}&p_{22}} \\
    trace(P) = p_{11} + p_{22}
\end{align}
Let $\mathcal{B}=\cbrak{e_{11}, e_{12}, e_{21}, e_{22}}$ be the ordered basis of V where, 
\begin{align}
    e_{11}=\myvec{1&0\\0&0}, e_{12}=\myvec{0&1\\0&0}\\
    e_{21}=\myvec{0&0\\1&0}, e_{22}=\myvec{0&0\\0&1}
\end{align}
Given, $T(A)=PA$, 
\begin{align}
    T(e_{11}) = \myvec{p_{11}&p_{12}\\ p_{21}&p_{22}}\myvec{1&0\\0&0}
    = \myvec{p_{11}&0\\p_{21}&0} \\
    = p_{11}e_{11} + p_{21}e_{21} \\
    T(e_{12}) = \myvec{p_{11}&p_{12}\\ p_{21}&p_{22}}\myvec{0&1\\0&0} = \myvec{0&p_{11}\\0&p_{21}}\\
    = p_{11}e_{12} + p_{21}e_{22} \\
    T(e_{21}) = \myvec{p_{11}&p_{12}\\ p_{21}&p_{22}}\myvec{0&0\\1&0} = 
    \myvec{p_{12}&0\\p_{22}&0}\\
    = p_{12}e_{11} + p_{22}e_{21} \\
    T(e_{22}) = \myvec{p_{11}&p_{12}\\ p_{21}&p_{22}}\myvec{0&0\\0&1} = 
    \myvec{0&p_{12}\\0&p_{22}} \\
    = p_{12}e_{12} + p_{22}e_{22}
\end{align}
The matrix representation of linear functional T in the ordered basis $\mathcal{B}$ is given as,
\begin{align}
    T = \myvec{[T(e_{11})]_{\mathcal{B}} & [T(e_{12})]_{\mathcal{B}} & [T(e_{21})]_{\mathcal{B}} & [T(e_{22})]_{\mathcal{B}}}
\end{align}
\begin{align}
    T = \myvec{p_{11}&0&p_{12}&0\\0&p_{11}&0&p_{12}\\p_{21}&0&p_{22}&0\\0&p_{21}&0&p_{22}}
\end{align}
Thus, 
\begin{align}
    trace(T) = 2\brak{p_{11}+p_{22}}\\
    \therefore trace(T)= 2trace(P)
\end{align}


\end{document}