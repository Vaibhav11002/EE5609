\documentclass[journal,12pt,twocolumn]{IEEEtran}
%
\usepackage{setspace}
\usepackage{gensymb}
\usepackage{siunitx}
\usepackage{tkz-euclide} 
\usepackage{textcomp}
\usepackage{standalone}
\usetikzlibrary{calc}

%\doublespacing
\singlespacing

%\usepackage{graphicx}
%\usepackage{amssymb}
%\usepackage{relsize}
\usepackage[cmex10]{amsmath}
%\usepackage{amsthm}
%\interdisplaylinepenalty=2500
%\savesymbol{iint}
%\usepackage{txfonts}
%\restoresymbol{TXF}{iint}
%\usepackage{wasysym}
\usepackage{amsthm}
%\usepackage{iithtlc}
\usepackage{mathrsfs}
\usepackage{txfonts}
\usepackage{stfloats}
\usepackage{bm}
\usepackage{cite}
\usepackage{cases}
\usepackage{subfig}
%\usepackage{xtab}
\usepackage{longtable}
\usepackage{multirow}
%\usepackage{algorithm}
%\usepackage{algpseudocode}
\usepackage{enumitem}
\usepackage{mathtools}
\usepackage{steinmetz}
\usepackage{tikz}
\usepackage{circuitikz}
\usepackage{verbatim}
\usepackage{tfrupee}
\usepackage[breaklinks=true]{hyperref}
%\usepackage{stmaryrd}
\usepackage{tkz-euclide} % loads  TikZ and tkz-base
%\usetkzobj{all}
\usetikzlibrary{calc,math}
\usepackage{listings}
    \usepackage{color}                                            %%
    \usepackage{array}                                            %%
    \usepackage{longtable}                                        %%
    \usepackage{calc}                                             %%
    \usepackage{multirow}                                         %%
    \usepackage{hhline}                                           %%
    \usepackage{ifthen}                                           %%
  %optionally (for landscape tables embedded in another document): %%
    \usepackage{lscape}     
\usepackage{multicol}
\usepackage{chngcntr}
\usepackage{amsmath}
\usepackage{cleveref}
%\usepackage{enumerate}

%\usepackage{wasysym}
%\newcounter{MYtempeqncnt}
\DeclareMathOperator*{\Res}{Res}
%\renewcommand{\baselinestretch}{2}
\renewcommand\thesection{\arabic{section}}
\renewcommand\thesubsection{\thesection.\arabic{subsection}}
\renewcommand\thesubsubsection{\thesubsection.\arabic{subsubsection}}

\renewcommand\thesectiondis{\arabic{section}}
\renewcommand\thesubsectiondis{\thesectiondis.\arabic{subsection}}
\renewcommand\thesubsubsectiondis{\thesubsectiondis.\arabic{subsubsection}}

% correct bad hyphenation here
\hyphenation{op-tical net-works semi-conduc-tor}
\def\inputGnumericTable{}                                 %%

\lstset{
%language=C,
frame=single, 
breaklines=true,
columns=fullflexible
}
%\lstset{
%language=tex,
%frame=single, 
%breaklines=true
%}
\usepackage{graphicx}
\usepackage{pgfplots}

\begin{document}
%


\newtheorem{theorem}{Theorem}[section]
\newtheorem{problem}{Problem}
\newtheorem{proposition}{Proposition}[section]
\newtheorem{lemma}{Lemma}[section]
\newtheorem{corollary}[theorem]{Corollary}
\newtheorem{example}{Example}[section]
\newtheorem{definition}[problem]{Definition}
%\newtheorem{thm}{Theorem}[section] 
%\newtheorem{defn}[thm]{Definition}
%\newtheorem{algorithm}{Algorithm}[section]
%\newtheorem{cor}{Corollary}
\newcommand{\BEQA}{\begin{eqnarray}}
\newcommand{\EEQA}{\end{eqnarray}}
\newcommand{\define}{\stackrel{\triangle}{=}}
\bibliographystyle{IEEEtran}
%\bibliographystyle{ieeetr}
\providecommand{\mbf}{\mathbf}
\providecommand{\pr}[1]{\ensuremath{\Pr\left(#1\right)}}
\providecommand{\qfunc}[1]{\ensuremath{Q\left(#1\right)}}
\providecommand{\sbrak}[1]{\ensuremath{{}\left[#1\right]}}
\providecommand{\lsbrak}[1]{\ensuremath{{}\left[#1\right.}}
\providecommand{\rsbrak}[1]{\ensuremath{{}\left.#1\right]}}
\providecommand{\brak}[1]{\ensuremath{\left(#1\right)}}
\providecommand{\lbrak}[1]{\ensuremath{\left(#1\right.}}
\providecommand{\rbrak}[1]{\ensuremath{\left.#1\right)}}
\providecommand{\cbrak}[1]{\ensuremath{\left\{#1\right\}}}
\providecommand{\lcbrak}[1]{\ensuremath{\left\{#1\right.}}
\providecommand{\rcbrak}[1]{\ensuremath{\left.#1\right\}}}
\theoremstyle{remark}
\newtheorem{rem}{Remark}
\newcommand{\sgn}{\mathop{\mathrm{sgn}}}
\providecommand{\abs}[1]{\left\vert#1\right\vert}
\providecommand{\res}[1]{\Res\displaylimits_{#1}} 
\providecommand{\norm}[1]{\left\lVert#1\right\rVert}
%\providecommand{\norm}[1]{\lVert#1\rVert}
\providecommand{\mtx}[1]{\mathbf{#1}}
\providecommand{\mean}[1]{E\left[ #1 \right]}
\providecommand{\fourier}{\overset{\mathcal{F}}{ \rightleftharpoons}}
%\providecommand{\hilbert}{\overset{\mathcal{H}}{ \rightleftharpoons}}
\providecommand{\system}{\overset{\mathcal{H}}{ \longleftrightarrow}}
	%\newcommand{\solution}[2]{\textbf{Solution:}{#1}}
\newcommand{\solution}{\noindent \textbf{Solution: }}
\newcommand{\cosec}{\,\text{cosec}\,}
\providecommand{\dec}[2]{\ensuremath{\overset{#1}{\underset{#2}{\gtrless}}}}
\newcommand{\myvec}[1]{\ensuremath{\begin{pmatrix}#1\end{pmatrix}}}
\newcommand{\mydet}[1]{\ensuremath{\begin{vmatrix}#1\end{vmatrix}}}
%\numberwithin{equation}{section}
\numberwithin{equation}{subsection}
%\numberwithin{problem}{section}
%\numberwithin{definition}{section}
\makeatletter
\@addtoreset{figure}{problem}
\makeatother
\let\StandardTheFigure\thefigure
\let\vec\mathbf
%\renewcommand{\thefigure}{\theproblem.\arabic{figure}}
\renewcommand{\thefigure}{\theproblem}
%\setlist[enumerate,1]{before=\renewcommand\theequation{\theenumi.\arabic{equation}}
%\counterwithin{equation}{enumi}
%\renewcommand{\theequation}{\arabic{subsection}.\arabic{equation}}
\def\putbox#1#2#3{\makebox[0in][l]{\makebox[#1][l]{}\raisebox{\baselineskip}[0in][0in]{\raisebox{#2}[0in][0in]{#3}}}}
     \def\rightbox#1{\makebox[0in][r]{#1}}
     \def\centbox#1{\makebox[0in]{#1}}
     \def\topbox#1{\raisebox{-\baselineskip}[0in][0in]{#1}}
     \def\midbox#1{\raisebox{-0.5\baselineskip}[0in][0in]{#1}}
\vspace{3cm}
\title{Assignment 4}
\author{Gaydhane Vaibhav Digraj \\ RollNo : AI20MTECH11002}
\maketitle
\newpage
%\tableofcontents
\bigskip
\renewcommand{\thefigure}{\theenumi}
\renewcommand{\thetable}{\theenumi}

\begin{abstract}
This document involves solving geometry concepts using linear algebra.
\end{abstract}
Download all latex-tikz codes from 
%
\begin{lstlisting}
https://github.com/Vaibhav11002/EE5609/tree/master/Assignment_4
\end{lstlisting}
%
\section{Problem}
$AB$ is a line segment and line $l$ is its perpendicular bisector. If a point $P$ lies on $l$, show
that $P$ is equidistant from $A$ and $B$. 

\section{Solution}

We have to prove that $P$ is equidistant from $A$ and $B$ i.e. length of lines $AP$ and $BP$ are equal. 

Let $DP$ be the perpendicular bisector of line $AB$. So,
\begin{align}
    \vec{A}-\vec{D} = \vec{D} - \vec{B} \\
    \norm{\vec{A}-\vec{D}} = \norm{\vec{D} - \vec{B}} = k \\
    \norm{\vec{D}-\vec{P}} = l
\end{align}

\renewcommand{\thefigure}{1}
\begin{figure}[!ht]
\centering
\resizebox{\columnwidth}{!}{\input{triangle_1.tex}}
\caption{$PD \perp AB$ by Latex-Tikz}
\label{fig:perp_bisector}
\end{figure}


Finding the length of line $AP$, 
\begin{multline}
    (\vec{A}-\vec{P})^T(\vec{A}-\vec{P}) = (\vec{A}- \vec{D}+\vec{D}-\vec{P})^T(\vec{A}-\vec{D}+\vec{D}-\vec{P}) \\
    =[(\vec{A}-\vec{D})^T+(\vec{D}-\vec{P})^T][(\vec{A}-\vec{D})+(\vec{D}-\vec{P})] \\
    =(\vec{A}-\vec{D})^T(\vec{A}-\vec{D}) + (\vec{A}-\vec{D})^T(\vec{D}-\vec{P}) + \\ (\vec{D}-\vec{P})^T(\vec{A}-\vec{D}) + (\vec{D}-\vec{P})^T(\vec{D}-\vec{P})
\end{multline}

Since, line $AB$ is perpendicular to line $DP$ the inner product is zero. 
\begin{align}
    (\vec{A}-\vec{D})^T(\vec{D}-\vec{P}) = (\vec{D}-\vec{P})^T(\vec{A}-\vec{D}) = 0
\end{align}
Thus, 
\begin{multline} \label{eq:len_ap}
    (\vec{A}-\vec{P})^T(\vec{A}-\vec{P}) = (\vec{A}-\vec{D})^T(\vec{A}-\vec{D}) + (\vec{D}-\vec{P})^T(\vec{D}-\vec{P}) \\
  \implies\norm{\vec{A}-\vec{P}}^2 = \norm{\vec{A}-\vec{D}}^2 + \norm{\vec{D}-\vec{P}}^2 \\
  \implies\norm{\vec{A}-\vec{P}} = \sqrt{k^2+l^2}
\end{multline}

Next finding the length of line $BP$,
\begin{multline}
    (\vec{B}-\vec{P})^T(\vec{B}-\vec{P}) = (\vec{B}- \vec{D}+\vec{D}-\vec{P})^T(\vec{B}-\vec{D}+\vec{D}-\vec{P}) \\
    =[(\vec{B}-\vec{D})^T+(\vec{D}-\vec{P})^T][(\vec{B}-\vec{D})+(\vec{D}-\vec{P})] \\
    =(\vec{B}-\vec{D})^T(\vec{B}-\vec{D}) + (\vec{B}-\vec{D})^T(\vec{D}-\vec{P}) + \\ (\vec{D}-\vec{P})^T(\vec{B}-\vec{D}) + (\vec{D}-\vec{P})^T(\vec{D}-\vec{P})
\end{multline}
Again since the inner product of lines $AB$ and $DP$ is zero, 
\begin{multline} \label{eq:len_bp}
  \implies(\vec{B}-\vec{P})^T(\vec{B}-\vec{P}) = (\vec{B}-\vec{D})^T(\vec{B}-\vec{D}) + (\vec{D}-\vec{P})^T(\vec{D}-\vec{P}) \\
  \implies\norm{\vec{B}-\vec{P}}^2 = \norm{\vec{B}-\vec{D}}^2 + \norm{\vec{D}-\vec{P}}^2 \\
  \implies\norm{\vec{B}-\vec{P}} = \sqrt{k^2+l^2}
\end{multline}

From equations \eqref{eq:len_ap} and \eqref{eq:len_bp} we get, 
\begin{align}
    \norm{\vec{A}-\vec{P}} = \norm{\vec{B}-\vec{P}}
\end{align}
Lengths of line $AP$ and $BP$ are equal. Hence, $P$ is equidistant from $A$ and $B$. 

\end{document}