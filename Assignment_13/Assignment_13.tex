\documentclass[journal,12pt,twocolumn]{IEEEtran}

\usepackage{setspace}
\usepackage{gensymb}

\singlespacing


\usepackage[cmex10]{amsmath}

\usepackage{amsthm}

\usepackage{mathrsfs}
\usepackage{txfonts}
\usepackage{stfloats}
\usepackage{bm}
\usepackage{cite}
\usepackage{cases}
\usepackage{subfig}

\usepackage{longtable}
\usepackage{multirow}

\usepackage{enumitem}
\usepackage{mathtools}
\usepackage{steinmetz}
\usepackage{tikz}
\usepackage{circuitikz}
\usepackage{verbatim}
\usepackage{tfrupee}
\usepackage[breaklinks=true]{hyperref}

\usepackage{tkz-euclide}

\usetikzlibrary{calc,math}
\usepackage{listings}
    \usepackage{color}                                            %%
    \usepackage{array}                                            %%
    \usepackage{longtable}                                        %%
    \usepackage{calc}                                             %%
    \usepackage{multirow}                                         %%
    \usepackage{hhline}                                           %%
    \usepackage{ifthen}                                           %%
    \usepackage{lscape}     
\usepackage{multicol}
\usepackage{chngcntr}

\DeclareMathOperator*{\Res}{Res}

\renewcommand\thesection{\arabic{section}}
\renewcommand\thesubsection{\thesection.\arabic{subsection}}
\renewcommand\thesubsubsection{\thesubsection.\arabic{subsubsection}}

\renewcommand\thesectiondis{\arabic{section}}
\renewcommand\thesubsectiondis{\thesectiondis.\arabic{subsection}}
\renewcommand\thesubsubsectiondis{\thesubsectiondis.\arabic{subsubsection}}


\hyphenation{op-tical net-works semi-conduc-tor}
\def\inputGnumericTable{}                                 %%

\lstset{
%language=C,
frame=single, 
breaklines=true,
columns=fullflexible
}
\begin{document}


\newtheorem{theorem}{Theorem}[section]
\newtheorem{problem}{Problem}
\newtheorem{proposition}{Proposition}[section]
\newtheorem{lemma}{Lemma}[section]
\newtheorem{corollary}[theorem]{Corollary}
\newtheorem{example}{Example}[section]
\newtheorem{definition}[problem]{Definition}

\newcommand{\BEQA}{\begin{eqnarray}}
\newcommand{\EEQA}{\end{eqnarray}}
\newcommand{\define}{\stackrel{\triangle}{=}}
\bibliographystyle{IEEEtran}
\providecommand{\mbf}{\mathbf}
\providecommand{\pr}[1]{\ensuremath{\Pr\left(#1\right)}}
\providecommand{\qfunc}[1]{\ensuremath{Q\left(#1\right)}}
\providecommand{\sbrak}[1]{\ensuremath{{}\left[#1\right]}}
\providecommand{\lsbrak}[1]{\ensuremath{{}\left[#1\right.}}
\providecommand{\rsbrak}[1]{\ensuremath{{}\left.#1\right]}}
\providecommand{\brak}[1]{\ensuremath{\left(#1\right)}}
\providecommand{\lbrak}[1]{\ensuremath{\left(#1\right.}}
\providecommand{\rbrak}[1]{\ensuremath{\left.#1\right)}}
\providecommand{\cbrak}[1]{\ensuremath{\left\{#1\right\}}}
\providecommand{\lcbrak}[1]{\ensuremath{\left\{#1\right.}}
\providecommand{\rcbrak}[1]{\ensuremath{\left.#1\right\}}}
\theoremstyle{remark}
\newtheorem{rem}{Remark}
\newcommand{\sgn}{\mathop{\mathrm{sgn}}}
\providecommand{\abs}[1]{\left\vert#1\right\vert}
\providecommand{\res}[1]{\Res\displaylimits_{#1}} 
\providecommand{\norm}[1]{\left\lVert#1\right\rVert}
%\providecommand{\norm}[1]{\lVert#1\rVert}
\providecommand{\mtx}[1]{\mathbf{#1}}
\providecommand{\mean}[1]{E\left[ #1 \right]}
\providecommand{\fourier}{\overset{\mathcal{F}}{ \rightleftharpoons}}
%\providecommand{\hilbert}{\overset{\mathcal{H}}{ \rightleftharpoons}}
\providecommand{\system}{\overset{\mathcal{H}}{ \longleftrightarrow}}
	%\newcommand{\solution}[2]{\textbf{Solution:}{#1}}
\newcommand{\solution}{\noindent \textbf{Solution: }}
\newcommand{\cosec}{\,\text{cosec}\,}
\providecommand{\dec}[2]{\ensuremath{\overset{#1}{\underset{#2}{\gtrless}}}}
\newcommand{\myvec}[1]{\ensuremath{\begin{pmatrix}#1\end{pmatrix}}}
\newcommand{\mydet}[1]{\ensuremath{\begin{vmatrix}#1\end{vmatrix}}}
\numberwithin{equation}{subsection}
\makeatletter
\@addtoreset{figure}{problem}
\makeatother
\let\StandardTheFigure\thefigure
\let\vec\mathbf
\renewcommand{\thefigure}{\theproblem}
\def\putbox#1#2#3{\makebox[0in][l]{\makebox[#1][l]{}\raisebox{\baselineskip}[0in][0in]{\raisebox{#2}[0in][0in]{#3}}}}
     \def\rightbox#1{\makebox[0in][r]{#1}}
     \def\centbox#1{\makebox[0in]{#1}}
     \def\topbox#1{\raisebox{-\baselineskip}[0in][0in]{#1}}
     \def\midbox#1{\raisebox{-0.5\baselineskip}[0in][0in]{#1}}
\vspace{3cm}
\title{Assignment 13}
\author{Gaydhane Vaibhav Digraj \\ Roll No. AI20MTECH11002}
\maketitle
\newpage
\bigskip
\renewcommand{\thefigure}{\theenumi}
\renewcommand{\thetable}{\theenumi}
\begin{abstract}
This document solves a problem involving basis and dimensions.
\end{abstract}
%
Download latex-tikz codes from 
%
\begin{lstlisting}
https://github.com/Vaibhav11002/EE5609/tree/master/Assignment_13
\end{lstlisting}
%
\section{Problem}
Let V be the vector space of all 2$\times$2 matrices over the field $\mathbb{F}$. Let $W_1$ be the set of matrices of the form 
\begin{align}
    \myvec{x&-x\\y&z}
\end{align}
and let $W_2$ be the set of matrices of the form 
\begin{align}
    \myvec{a&b\\-a&c}
\end{align}
\begin{enumerate}
    \item Prove that $W_1$ and $W_2$ are subspaces of V.
    \item Find the dimension of $W_1, W_2, W_1+W_2$ and $W_1\cap W_2$.
\end{enumerate}

\section{Theorem}
A non-empty subset W of V is a subspace of V if and only if for each pair of vectors $\alpha$, $\beta$ in W and each scalar c $\in$ F, the vector $c\alpha+\beta \in$ W.

\section{Solution}
\begin{enumerate}
\item Let ${A_1,A_2}\in W_1$ where,
\begin{align}
    A_1 = \myvec{x_1&-x_1\\y_1&z_1}, A_2 = \myvec{x_2&-x_2\\y_2&z_2}
\end{align}
Let $c\in F$ then,
\begin{align}
    cA_1+A_2 = \myvec{cx_1+x_2&-cx_1-x_2\\cy_1+y_2&cz_1+z_2} = \myvec{u&-u\\v&w}
\end{align}
Thus $cA_1+A_2 \in W_1$. \textbf{Hence $W_1$ is a subspace.} 
Similarly, let ${A_1,A_2}\in W_2$ where,
\begin{align}
    A_1 = \myvec{a_1&b_1\\-a_1&c_1}, A_2 = \myvec{a_2&b_2\\-a_2&c_2}
\end{align}
Let $c\in F$ then,
\begin{align}
    cA_1+A_2 = \myvec{ca_1+a_2&cb_1+b_2\\-ca_1-a_2&cc_1+c_2} = \myvec{u&v\\-u&w}
\end{align}
Thus $cA_1+A_2 \in W_2$. \textbf{Hence $W_2$ is a subspace.}

\item The subspace $W_1$ can be given as, 
\begin{align}
    \myvec{x&-x\\y&z} &= x\myvec{1&-1\\0&0} + y\myvec{0&0\\1&0} + z\myvec{0&0\\0&1} \\
    &= xA_1 + yA_2 + zA_2 \label{span_w1}
\end{align}
Now, 
\begin{align}
    &x\myvec{1&-1\\0&0} + y\myvec{0&0\\1&0} + z\myvec{0&0\\0&1} = \myvec{0&0\\0&0} \\
    &\implies x = y = z = 0 \label{indep_w1}
\end{align}
$A_1, A_2, A_3$ are linearly independent and spans $W_1$. Thus $\cbrak{A_1, A_2, A_3}$ forms basis for $W_1$. 

\textbf{$\therefore$ dimension of $W_1$ is 3.}   

The subspace $W_2$ can be given as, 
\begin{align}
    \myvec{a&b\\-a&c} &= a\myvec{1&0\\-1&0} + b\myvec{0&1\\0&0} + c\myvec{0&0\\0&1} \\
    &= aA_1 + bA_2 + cA_2 \label{span_w2}
\end{align}
Now, 
\begin{align}
    &a\myvec{1&0\\-1&0} + b\myvec{0&1\\0&0} + c\myvec{0&0\\0&1} = \myvec{0&0\\0&0} \\
    &\implies a = b = c = 0 \label{indep_w2}
\end{align}
$A_1, A_2, A_3$ are linearly independent and spans $W_2$. Thus $\cbrak{A_1, A_2, A_3}$ forms basis for $W_2$. 

\textbf{$\therefore$ dimension of $W_2$ is 3.}

Subspace $W_1+W_2$ is given by, 
\begin{align}
    &\myvec{x+a&-x+b\\y-a&z+c}
\end{align}
For $x+a\ne -x+b\ne y-a\ne z+c $, 
\begin{align}
    &\myvec{x+a&-x+b\\y-a&z+c} = \myvec{j&k\\l&m} \\
    &= j\myvec{1&0\\0&0}+k\myvec{0&1\\0&0}+l\myvec{0&0\\1&0} + m\myvec{0&0\\0&1} \\
    &= jA_1 + kA_2 + lA_3 + mA_4
\end{align}
Now, 
\begin{align}
    &jA_1 + kA_2 + lA_3 + mA_4 = 0 \\
    &\implies j=k=l=m=0
\end{align}
$A_1, A_2, A_3, A_4 $ are linearly independent and spans $W_1+W_2$. 
Thus $\cbrak{A_1, A_2, A_3, A_4}$ forms a basis. 

\textbf{$\therefore$ dimension of $W_1+W_2$ is 4.}

The subspace $W_1 \cap W_2$ is given as, 
\begin{align}
   \myvec{x&-x\\-x&y} &= x\myvec{1&-1\\-1&0} + y\myvec{0&0\\0&1} \\
   &= xA_1 + yA_2
\end{align}
 Now, 
\begin{align}
    &x\myvec{1&-1\\-1&0} + y\myvec{0&0\\0&1} = \myvec{0&0\\0&0} \\
    &\implies x = y = 0
\end{align}
$A_1, A_2$ are linearly independent and spans $W_1 \cap W_2$. Thus, $\cbrak{A_1, A_2}$ forms a basis. 

\textbf{$\therefore$ dimension of $W_1 \cap W_2$ is 2.}
    
    



\end{enumerate}
\end{document}