\documentclass[journal,12pt,twocolumn]{IEEEtran}

\usepackage{setspace}
\usepackage{gensymb}

\singlespacing


\usepackage[cmex10]{amsmath}

\usepackage{amsthm}

\usepackage{mathrsfs}
\usepackage{txfonts}
\usepackage{stfloats}
\usepackage{bm}
\usepackage{cite}
\usepackage{cases}
\usepackage{subfig}

\usepackage{longtable}
\usepackage{multirow}

\usepackage{enumitem}
\usepackage{mathtools}
\usepackage{steinmetz}
\usepackage{tikz}
\usepackage{circuitikz}
\usepackage{verbatim}
\usepackage{tfrupee}
\usepackage[breaklinks=true]{hyperref}

\usepackage{tkz-euclide}

\usetikzlibrary{calc,math}
\usepackage{listings}
    \usepackage{color}                                            %%
    \usepackage{array}                                            %%
    \usepackage{longtable}                                        %%
    \usepackage{calc}                                             %%
    \usepackage{multirow}                                         %%
    \usepackage{hhline}                                           %%
    \usepackage{ifthen}                                           %%
    \usepackage{lscape}     
\usepackage{multicol}
\usepackage{chngcntr}

\DeclareMathOperator*{\Res}{Res}

\renewcommand\thesection{\arabic{section}}
\renewcommand\thesubsection{\thesection.\arabic{subsection}}
\renewcommand\thesubsubsection{\thesubsection.\arabic{subsubsection}}

\renewcommand\thesectiondis{\arabic{section}}
\renewcommand\thesubsectiondis{\thesectiondis.\arabic{subsection}}
\renewcommand\thesubsubsectiondis{\thesubsectiondis.\arabic{subsubsection}}


\hyphenation{op-tical net-works semi-conduc-tor}
\def\inputGnumericTable{}                                 %%

\lstset{
%language=C,
frame=single, 
breaklines=true,
columns=fullflexible
}
\begin{document}


\newtheorem{theorem}{Theorem}[section]
\newtheorem{problem}{Problem}
\newtheorem{proposition}{Proposition}[section]
\newtheorem{lemma}{Lemma}[section]
\newtheorem{corollary}[theorem]{Corollary}
\newtheorem{example}{Example}[section]
\newtheorem{definition}[problem]{Definition}

\newcommand{\BEQA}{\begin{eqnarray}}
\newcommand{\EEQA}{\end{eqnarray}}
\newcommand{\define}{\stackrel{\triangle}{=}}
\bibliographystyle{IEEEtran}
\providecommand{\mbf}{\mathbf}
\providecommand{\pr}[1]{\ensuremath{\Pr\left(#1\right)}}
\providecommand{\qfunc}[1]{\ensuremath{Q\left(#1\right)}}
\providecommand{\sbrak}[1]{\ensuremath{{}\left[#1\right]}}
\providecommand{\lsbrak}[1]{\ensuremath{{}\left[#1\right.}}
\providecommand{\rsbrak}[1]{\ensuremath{{}\left.#1\right]}}
\providecommand{\brak}[1]{\ensuremath{\left(#1\right)}}
\providecommand{\lbrak}[1]{\ensuremath{\left(#1\right.}}
\providecommand{\rbrak}[1]{\ensuremath{\left.#1\right)}}
\providecommand{\cbrak}[1]{\ensuremath{\left\{#1\right\}}}
\providecommand{\lcbrak}[1]{\ensuremath{\left\{#1\right.}}
\providecommand{\rcbrak}[1]{\ensuremath{\left.#1\right\}}}
\theoremstyle{remark}
\newtheorem{rem}{Remark}
\newcommand{\sgn}{\mathop{\mathrm{sgn}}}
\providecommand{\abs}[1]{\left\vert#1\right\vert}
\providecommand{\res}[1]{\Res\displaylimits_{#1}} 
\providecommand{\norm}[1]{\left\lVert#1\right\rVert}
%\providecommand{\norm}[1]{\lVert#1\rVert}
\providecommand{\mtx}[1]{\mathbf{#1}}
\providecommand{\mean}[1]{E\left[ #1 \right]}
\providecommand{\fourier}{\overset{\mathcal{F}}{ \rightleftharpoons}}
%\providecommand{\hilbert}{\overset{\mathcal{H}}{ \rightleftharpoons}}
\providecommand{\system}{\overset{\mathcal{H}}{ \longleftrightarrow}}
	%\newcommand{\solution}[2]{\textbf{Solution:}{#1}}
\newcommand{\solution}{\noindent \textbf{Solution: }}
\newcommand{\cosec}{\,\text{cosec}\,}
\providecommand{\dec}[2]{\ensuremath{\overset{#1}{\underset{#2}{\gtrless}}}}
\newcommand{\myvec}[1]{\ensuremath{\begin{pmatrix}#1\end{pmatrix}}}
\newcommand{\mydet}[1]{\ensuremath{\begin{vmatrix}#1\end{vmatrix}}}
\numberwithin{equation}{subsection}
\makeatletter
\@addtoreset{figure}{problem}
\makeatother
\let\StandardTheFigure\thefigure
\let\vec\mathbf
\renewcommand{\thefigure}{\theproblem}
\def\putbox#1#2#3{\makebox[0in][l]{\makebox[#1][l]{}\raisebox{\baselineskip}[0in][0in]{\raisebox{#2}[0in][0in]{#3}}}}
     \def\rightbox#1{\makebox[0in][r]{#1}}
     \def\centbox#1{\makebox[0in]{#1}}
     \def\topbox#1{\raisebox{-\baselineskip}[0in][0in]{#1}}
     \def\midbox#1{\raisebox{-0.5\baselineskip}[0in][0in]{#1}}
\vspace{3cm}
\title{Assignment 7}
\author{Gaydhane Vaibhav Digraj \\ Roll No. AI20MTECH11002}
\maketitle
\newpage
\bigskip
\renewcommand{\thefigure}{\theenumi}
\renewcommand{\thetable}{\theenumi}
\begin{abstract}
This document performs the QR decomposition on a matrix. 
\end{abstract}
%
Download latex-tikz codes from 
%
\begin{lstlisting}
https://github.com/Vaibhav11002/EE5609/tree/master/Assignment_7
\end{lstlisting}
%
\section{Problem}
Find the QR decomposition of $\myvec{3&2\\1&4}$ 

\section{Solution}

Let $\vec{c_1}$ and $\vec{c_2}$ be the column vectors of the given matrix.
\begin{align}
    \vec{c_1} &= \myvec{3\\1} \label{eq:1}\\
    \vec{c_2} &= \myvec{2\\4} \label{eq:2} 
\end{align}

The column vectors can be represented as,
\begin{align}
\vec{c_1} &=k_1\vec{u}_1 \label{eq:3}\\
\vec{c_2} &=r_1\vec{u}_1+k_2\vec{u}_2 \label{eq:4}
\end{align}
where, 
\begin{align}
k_1 &= \norm{\vec{c_1}}\label{eq:5}\\
\vec{u_1} &= \frac{\vec{c_1}}{k_1} \label{eq:6}\\
r_1 &= \frac{\vec{u}_1^T\vec{c_2}}{\norm{\vec{u}_1}^2} \label{eq:7}\\
\vec{u_2} &= \frac{\vec{c_2} - r_1 \vec{u_1}}{\norm{\vec{c_2} - r_1 \vec{u_1}}} \label{eq:8}\\
k_2 &= {\vec{u_2}^T\vec{c_2}}\label{eq:9}
\end{align}

From \eqref{eq:3} and \eqref{eq:4}, 
\begin{align}
\myvec{\vec{c_1} & \vec{c_2}} &= \myvec{\vec{u}_1 & \vec{u}_2}\myvec{k_1 & r_1 \\ 0 & k_2} \\
\myvec{\vec{c_1} & \vec{c_2}} &= \vec{Q}\vec{R}
\end{align}
Where $\vec{R}$ is an upper triangular matrix and
\begin{align}
\vec{Q}^T\vec{Q}&=\vec{I}
\end{align}
Using equations \eqref{eq:5} to \eqref{eq:9} we get, 
\begin{align}
    k_1 &= \sqrt{3^2+1^2}=\sqrt{10} \label{eq:10}\\
    \vec{u_1} &= \frac{1}{\sqrt{10}}\myvec{3\\1}= \myvec{\frac{3}{\sqrt{10}} \\ \frac{1}{\sqrt{10}}} \\
    r_1 &= \myvec{\frac{3}{\sqrt{10}} & \frac{1}{\sqrt{10}}}\myvec{2\\4}=\sqrt{10} \\
    \vec{u_2} &= \myvec{\frac{-1}{\sqrt{10}} \\ \frac{3}{\sqrt{10}}} \\
    k_2 &= \myvec{\frac{-1}{\sqrt{10}} & \frac{3}{\sqrt{10}}}\myvec{2\\4} = \sqrt{10} \label{eq:11}
\end{align}
Now putting the values from \eqref{eq:10} to \eqref{eq:11}, we obtain the QR decomposition of given matrix, 
\begin{align}
    \myvec{3&2\\1&4} &= \myvec{\frac{3}{\sqrt{10}}&\frac{-1}{\sqrt{10}}\\\frac{1}{\sqrt{10}}&\frac{3}{\sqrt{10}}}\myvec{\sqrt{10}&\sqrt{10}\\0&\sqrt{10}}
\end{align}

\end{document}