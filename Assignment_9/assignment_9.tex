\documentclass[journal,12pt,twocolumn]{IEEEtran}

\usepackage{setspace}
\usepackage{gensymb}

\singlespacing


\usepackage[cmex10]{amsmath}

\usepackage{amsthm}

\usepackage{mathrsfs}
\usepackage{txfonts}
\usepackage{stfloats}
\usepackage{bm}
\usepackage{cite}
\usepackage{cases}
\usepackage{subfig}

\usepackage{longtable}
\usepackage{multirow}

\usepackage{enumitem}
\usepackage{mathtools}
\usepackage{steinmetz}
\usepackage{tikz}
\usepackage{circuitikz}
\usepackage{verbatim}
\usepackage{tfrupee}
\usepackage[breaklinks=true]{hyperref}

\usepackage{tkz-euclide}

\usetikzlibrary{calc,math}
\usepackage{listings}
    \usepackage{color}                                            %%
    \usepackage{array}                                            %%
    \usepackage{longtable}                                        %%
    \usepackage{calc}                                             %%
    \usepackage{multirow}                                         %%
    \usepackage{hhline}                                           %%
    \usepackage{ifthen}                                           %%
    \usepackage{lscape}     
\usepackage{multicol}
\usepackage{chngcntr}

\DeclareMathOperator*{\Res}{Res}

\renewcommand\thesection{\arabic{section}}
\renewcommand\thesubsection{\thesection.\arabic{subsection}}
\renewcommand\thesubsubsection{\thesubsection.\arabic{subsubsection}}

\renewcommand\thesectiondis{\arabic{section}}
\renewcommand\thesubsectiondis{\thesectiondis.\arabic{subsection}}
\renewcommand\thesubsubsectiondis{\thesubsectiondis.\arabic{subsubsection}}


\hyphenation{op-tical net-works semi-conduc-tor}
\def\inputGnumericTable{}                                 %%

\lstset{
%language=C,
frame=single, 
breaklines=true,
columns=fullflexible
}
\begin{document}


\newtheorem{theorem}{Theorem}[section]
\newtheorem{problem}{Problem}
\newtheorem{proposition}{Proposition}[section]
\newtheorem{lemma}{Lemma}[section]
\newtheorem{corollary}[theorem]{Corollary}
\newtheorem{example}{Example}[section]
\newtheorem{definition}[problem]{Definition}

\newcommand{\BEQA}{\begin{eqnarray}}
\newcommand{\EEQA}{\end{eqnarray}}
\newcommand{\define}{\stackrel{\triangle}{=}}
\bibliographystyle{IEEEtran}
\providecommand{\mbf}{\mathbf}
\providecommand{\pr}[1]{\ensuremath{\Pr\left(#1\right)}}
\providecommand{\qfunc}[1]{\ensuremath{Q\left(#1\right)}}
\providecommand{\sbrak}[1]{\ensuremath{{}\left[#1\right]}}
\providecommand{\lsbrak}[1]{\ensuremath{{}\left[#1\right.}}
\providecommand{\rsbrak}[1]{\ensuremath{{}\left.#1\right]}}
\providecommand{\brak}[1]{\ensuremath{\left(#1\right)}}
\providecommand{\lbrak}[1]{\ensuremath{\left(#1\right.}}
\providecommand{\rbrak}[1]{\ensuremath{\left.#1\right)}}
\providecommand{\cbrak}[1]{\ensuremath{\left\{#1\right\}}}
\providecommand{\lcbrak}[1]{\ensuremath{\left\{#1\right.}}
\providecommand{\rcbrak}[1]{\ensuremath{\left.#1\right\}}}
\theoremstyle{remark}
\newtheorem{rem}{Remark}
\newcommand{\sgn}{\mathop{\mathrm{sgn}}}
\providecommand{\abs}[1]{\left\vert#1\right\vert}
\providecommand{\res}[1]{\Res\displaylimits_{#1}} 
\providecommand{\norm}[1]{\left\lVert#1\right\rVert}
%\providecommand{\norm}[1]{\lVert#1\rVert}
\providecommand{\mtx}[1]{\mathbf{#1}}
\providecommand{\mean}[1]{E\left[ #1 \right]}
\providecommand{\fourier}{\overset{\mathcal{F}}{ \rightleftharpoons}}
%\providecommand{\hilbert}{\overset{\mathcal{H}}{ \rightleftharpoons}}
\providecommand{\system}{\overset{\mathcal{H}}{ \longleftrightarrow}}
	%\newcommand{\solution}[2]{\textbf{Solution:}{#1}}
\newcommand{\solution}{\noindent \textbf{Solution: }}
\newcommand{\cosec}{\,\text{cosec}\,}
\providecommand{\dec}[2]{\ensuremath{\overset{#1}{\underset{#2}{\gtrless}}}}
\newcommand{\myvec}[1]{\ensuremath{\begin{pmatrix}#1\end{pmatrix}}}
\newcommand{\mydet}[1]{\ensuremath{\begin{vmatrix}#1\end{vmatrix}}}
\numberwithin{equation}{subsection}
\makeatletter
\@addtoreset{figure}{problem}
\makeatother
\let\StandardTheFigure\thefigure
\let\vec\mathbf
\renewcommand{\thefigure}{\theproblem}
\def\putbox#1#2#3{\makebox[0in][l]{\makebox[#1][l]{}\raisebox{\baselineskip}[0in][0in]{\raisebox{#2}[0in][0in]{#3}}}}
     \def\rightbox#1{\makebox[0in][r]{#1}}
     \def\centbox#1{\makebox[0in]{#1}}
     \def\topbox#1{\raisebox{-\baselineskip}[0in][0in]{#1}}
     \def\midbox#1{\raisebox{-0.5\baselineskip}[0in][0in]{#1}}
\vspace{3cm}
\title{Assignment 9}
\author{Gaydhane Vaibhav Digraj \\ Roll No. AI20MTECH11002}
\maketitle
\newpage
\bigskip
\renewcommand{\thefigure}{\theenumi}
\renewcommand{\thetable}{\theenumi}
\begin{abstract}
This document solves whether two system of linear equations are linear equivalent or not.  
\end{abstract}
%
Download latex-tikz codes from 
%
\begin{lstlisting}
https://github.com/Vaibhav11002/EE5609/tree/master/Assignment_9
\end{lstlisting}
%
\section{Problem}
Let $\mathbb{F}$ be the field of complex numbers. Are the following two systems of linear equations equivalent? If so, express each equation in each system as a linear combination of the equations in the other system.
\begin{align*}
    x_1 - x_2 &=0\\
    2x_1 + x_2 &=0
\end{align*}
and 
\begin{align*}
    3x_1 + x_2 &=0 \\
    x_1 + x_2 &=0
\end{align*}
\section{Solution}
The given system of linear equations can be written as,   
\begin{align}
    \vec{A}\vec{x} &= 0 \\
    \implies\myvec{1&-1\\2&1}\vec{x} &= 0 \label{ax=0}\\
    \vec{B}\vec{x} &= 0 \\
    \implies\myvec{3&1\\1&1}\vec{x} &= 0 \label{bx=0}
\end{align}
Now we can obtain $\vec{B}$ from matrix $\vec{A}$ by performing elementary row operations given as, 
\begin{align}
    &\vec{B} = \vec{C}\vec{A} \\
    &\myvec{3&1\\1&1} = \vec{C}\myvec{1&-1\\2&1}
\end{align}
where $\vec{C}$ is product of elementary matrices given as, 
\begin{multline}
    \vec{C} = \brak{\vec{E_7}\vec{E_6}\vec{E_5}\vec{E_4}\vec{E_3}\vec{E_2}\vec{E_1}}\\
    =\myvec{1&0\\\frac{1}{3}&1}\myvec{1&0\\0&\frac{2}{3}}\myvec{1&1\\0&1}\myvec{3&0\\0&1}\myvec{1&1\\0&1}\myvec{1&0\\0&\frac{1}{3}}\myvec{1&0\\-2&1} \\
    =\myvec{\frac{1}{3}&\frac{4}{3} \\ \frac{-1}{3}&\frac{2}{3}}
\end{multline}
Now, performing elementary operations on the right side of $\vec{A}$ we obtain matrix $\vec{B}$ given as, 
\begin{align}
    &\vec{B} = \vec{A}\vec{C}^{'} \\
    &\myvec{3&1\\1&1} = \myvec{1&-1\\2&1}\vec{C}^{'}
\end{align}
where, $\vec{C}^{'}$ is product of elementary matrices given by, 
\begin{multline}
    \vec{C}^{'} = \brak{\vec{E_1}\vec{E_2}\vec{E_3}\vec{E_4}\vec{E_5}}\\
    =\myvec{1&0\\-2&1}\myvec{\frac{1}{3}&0\\0&1}\myvec{1&2\\0&1}\myvec{2&0\\0&1}\myvec{1&0\\1&1}
    = \myvec{\frac{4}{3}&\frac{2}{3}\\ \frac{-5}{3}&\frac{-1}{3}}
\end{multline}
Thus \eqref{bx=0} can be obtained from \eqref{ax=0} by multiplying it with matrix $\vec{C}$ or $\vec{C}^{'}$, where each one is product of elementary matrices and hence invertible. 
Thus the two given homogeneous systems are row equivalent.  
Now writing equations in matrix-vector form as, 
\begin{align}
    3x_1 + x_2 &= \myvec{3&1}\vec{x} \\
    \implies \myvec{3&1}\vec{x} &= \frac{1}{3}\myvec{1&-1}\vec{x} + \frac{4}{3}\myvec{2&1}\vec{x} \label{b1}\\
    x_1 + x_2 &= \myvec{1&1}\vec{x} \\
    \implies \myvec{1&1}\vec{x} &= \frac{-1}{3}\myvec{1&-1}\vec{x} + 
    \frac{2}{3}\myvec{2&1}\vec{x} \label{b2}
\end{align}
Thus, each equation in second system can be expressed as linear combination of equations in first system. 
This can be verified from the entries of $\vec{C}$ as they take the linear combination of each rows of matrix $\vec{A}$. 
Similarly, 
\begin{align}
    x_1 - x_2 &= \myvec{1&-1}\vec{x} \\
    \implies\myvec{1&-1}\vec{x} &= (1)\myvec{3&1}\vec{x} + (-2)\myvec{1&1}\vec{x} \label{a1}\\
    2x_1 + x_2 &= \myvec{2&1}\vec{x} \\
    \implies \myvec{2&1}\vec{x} &= \frac{1}{2}\myvec{3&1}\vec{x} + \frac{1}{2}\myvec{1&1}\vec{x} \label{a2}
\end{align}
\end{document}
